
%--- Packages to use
%
\usepackage[]{fancyhdr}   
\usepackage[]{natbib}
\usepackage{alltt}
\usepackage{times}
\usepackage{lscape}         % landscape mode of a single page
\usepackage[]{longtable}    % allow tables longer than one page
\usepackage{makeidx}        % index of terms
\usepackage{tabularx}       % allows line breaking in table columns
\usepackage{algorithm}      % for describing algorithms with pseudo-code
\usepackage{algorithmic}
\usepackage{ifthen}
\usepackage{ifpdf}
\usepackage{xr-hyper}
\usepackage{fancyvrb}
\usepackage{xcolor}
\usepackage{amsmath}
\usepackage{minted}
\usepackage{wasysym}


%--- Margins
%
\voffset-1.5cm
\headheight16pt
\headsep1.1cm
\textheight23cm
\hoffset-1.3cm
\oddsidemargin2.2cm
\textwidth14.0cm


%--- Headings
%
\pagestyle{fancy}
\renewcommand{\chaptermark}[1]{\markboth{#1}{}}
\renewcommand{\sectionmark}[1]{\markright{\thesection\ #1}}
\fancyhf{}
\fancyhead[LE,RO]{\small{\sc\thepage}}
\fancyhead[LO]{\small{\scshape\rightmark}}
\fancyhead[RE]{\small{\scshape\leftmark}}
\renewcommand{\headrulewidth}{0.5pt}
\renewcommand{\footrulewidth}{0pt}
\fancypagestyle{plain}{%
  \fancyhead{}
  \renewcommand{\headrulewidth}{0pt}
}  


%--- Some layout commands
%
\sloppy
\raggedbottom
\hbadness=10000
\makeindex
\bibliographystyle{agu}


%--- Some fixes

%- To avoid hyperref error:
\newcommand{\theHalgorithm}{\theHchapter.\arabic{algorithm}}

%- Width of the caption in longtable:
\setlength{\LTcapwidth}{0.9\textwidth} 

%- Change brace type for comments in algorithmic
\renewcommand{\algorithmiccomment}[1]{(#1)}


%--- Symbol definitions
%
\input{symbol_defs.tex}



%--- PDF/LaTeX specific options
\ifpdf
  \usepackage{graphicx}    % includegraphics
  \DeclareGraphicsExtensions{.pdf}
  \usepackage{color}
  \definecolor{ArtsBlue}{RGB}{2, 113, 187}
  \definecolor{DarkRed}{rgb}{0.5,0,0}
  \usepackage
    [pdftex,                         % or dvips
     colorlinks=true,
     allcolors=ArtsBlue,
%     pdftitle={ARTS User Guide},
%     pdfauthor={The ARTS development team},
%     pdfsubject={},
%     pdfkeywords={},
%     bookmarks=true,
%     bookmarksopen=false,
%     pdfpagemode=None,
%     plainpages=false,
%     pdfpagelabels
      ]
  {hyperref}
  \setcounter{tocdepth}{3}
\else
  \usepackage{graphicx}    % includegraphics
  \DeclareGraphicsExtensions{.eps}
  \setcounter{tocdepth}{1}
\fi


%--- Command definitions -----------------------------------------------------

%- Document history
\newcommand{\starthistory} {\begin{table}[b]  \begin{tabular}{l p{11cm}} 
                             \hline {\bf History} & \\ }
\newcommand{\stophistory}  {\end{tabular} \end{table} }


%- Symbol table
\newcommand{\startsymbols} {\begin{table} \begin{center} 
                            \caption{Examples of symbols used in this chapter,
                            the corresponding notation in the ARTS source code
                            and a short description of the quantity. }
                            \begin{tabular}{l l l}
                            {\bf Here} & {\bf In ARTS} & {\bf Description} 
                            \\ \hline \\ } 
\newcommand{\stopsymbols}  {\\ \hline \end{tabular} 
                           \end{center} \end{table}}      
\newcommand{\startsymbolswithunits} 
                   {\begin{table} \begin{center} 
                            \caption{Examples of symbols used in this chapter,
                            the corresponding notation in the ARTS source code
                            and a short description of the quantity. }
                   \begin{tabular}{l l l l}
                   {\bf Here} & {\bf Unit} & {\bf In ARTS} & {\bf Description} 
                   \\ \hline \\ } 
\newcommand{\stopsymbolswithunits} {\stopsymbols}

% ARTS Web address
\newcommand{\artsserver}{https://www.radiativetransfer.org/}
\newcommand{\artsdocserver}{https://atmtools.github.io/arts-docs-master/}

% Docserver for this ARTS version
\newcommand{\docserver}{stubs/}

%- Command to link to a URL on the ARTS web server
% takes the subpath (without leading / !!!) as input
\newcommand{\artsurl}[1]{\href{\artsserver #1}{\artsserver #1}}
\newcommand{\artsdocsurl}[1]{\href{\artsdocserver #1}{\artsdocserver #1}}

%- Command to create link to ARTS built-in documentation. (Consider
% using \wsmindex, \wsvindex, etc., instead. They use this command
% implicitly.  But direct use may be useful if you use the same term
% several times in a short section, and don't want all of these
% occurrences to be in the index.)
% Underscores must be escaped by leading backslash!
\newcommand{\builtindoc}[1]{\href{\artsdocserver\docserver pyarts.workspace.Workspace.#1.html}{#1}}

%- Command to write an internal ARTS variable, internal function, or
% file name with special style. Anything that does not have built-in
% documentation. Also for other things that are code,
% but inside the text. Use the "code" environment for longer pieces of
% code.
% Underscores must be escaped by leading backslash!
\newcommand{\shortcode}[1]{\texttt{#1}}

%- Define verbatim environment for arts code examples.
% (For longer pieces of code, for in-text use "\shortcode".)
% This is the only code command where you do not have to escape
% underscores. 
\DefineVerbatimEnvironment{code}{Verbatim}{fontsize=\small}


%- Commands for easy indexing of terms
%
% Underscores must be escaped by leading backslash!
%
% Index command to use when text and index reference are equal. Otherwise
% the normal \index command must be used.
\newcommand{\textindex}[1]{#1\index{#1}} 
%
% Index command to make index for a workspace method. It writes out the
% function name in verbatim style and makes an index reference.
\newcommand{\wsmindex}[1]{\builtindoc{#1}\index{workspace methods!#1}} 
%
% Index command to make index for workspace variable. Works as \wsmindex.
\newcommand{\wsvindex}[1]{\builtindoc{#1}\index{workspace variables!#1}}
%
% Index command to make index for workspace agenda. Works as \wsmindex.
\newcommand{\wsaindex}[1]{\builtindoc{#1}\index{workspace agendas!#1}} 
%
% Index command to make index for a ARTS file. Works as \wsmindex.
\newcommand{\fileindex}[1]{\shortcode{#1}\index{ARTS files!#1}}
%
% Index command to make index for an internal function. Works as \wsmindex.
\newcommand{\funcindex}[1]{\shortcode{#1}\index{internal ARTS functions!#1}}
%
% Index command to make index for a ARTS data structure. Works as \wsmindex.
\newcommand{\typeindex}[1]{\shortcode{#1}\index{data types!#1}}


%- For FIXMEs:
\newcommand{\FIXME}[1]{\textcolor{gray}{\bfseries FIXME: #1}}


%- Names of the different documentation documents:
\newcommand{\user}{\emph{ARTS User Guide}}
\newcommand{\developer}{\emph{ARTS Developer Guide}}
\newcommand{\theory}{\emph{ARTS Theory}}

%- Cope comments
\usemintedstyle{sas}
\definecolor{LightGray}{gray}{0.9}
\setminted[cpp]{
frame=lines,
framesep=2mm,
baselinestretch=1.2,
fontsize=\footnotesize,
autogobble,
label=C++
}
\setminted[python]{
frame=lines,
framesep=2mm,
baselinestretch=1.2,
fontsize=\footnotesize,
autogobble,
label=Python,
python3=true
}
\setmintedinline{fontsize=\footnotesize,bgcolor={}}

%------------------------------------------------------------------------------



%%% Local Variables: 
%%% mode: latex
%%% TeX-master: t
%%% End: 
